\documentclass{article}  % 选择模版,这里是使用Latex自带的article模版
    \title{Numerical Computing Homework 3}
    \author{Shaosen Hou 18340055}
    \usepackage{multirow}   % 插入表格用到的宏包
    \usepackage{enumerate}
    \usepackage{threeparttable}
    \usepackage{amsmath,bm}
    \usepackage{amsfonts}
    \usepackage[noend]{algpseudocode}
    \usepackage{subcaption}
    \usepackage[english]{babel}	
    \usepackage{paralist}	
    \usepackage[lowtilde]{url}
    \usepackage{listings}
    \usepackage{color}
    \usepackage{hyperref}
    \usepackage{mathtools}
\begin{document} 
    \maketitle
        \subsection*{Page 124} 
        \paragraph{3.}Solve the upper-triangular system and find the value of the determinant of the coefficient matrix.
        \begin{align*}
            4x_1 - x_2 + 2x_3 + 2x_4 - x_5 &= 4 \\
            -2x_2 + 6x_3 + 2x_4 + 7x_5 &= 0 \\
            x_3 - x_4 - 2x_5 &= 3 \\
            -2x_4 - x_5 &= 10 \\
            3x_5 &= 6 \\
        \end{align*}
        \paragraph{Solution:}
        \paragraph{}We find that all the diagonal elements are non-zero. So we could solve the upper-triangular system by the method of back substitution.
        \paragraph{}The coefficient matrix $\mathbf{A}$ and the matrix $\mathbf{B}$ are: 
        \begin{align*}
            \mathbf{A} =  \begin{bmatrix*}[r]
                4 & -1 & 2 & 2 & -1 \\
                0 & -2 & 6 & 2 & 7 \\
                0 & 0 & 1 & -1 & -2 \\
                0 & 0 & 0 & -2 & -1 \\
                0 & 0 & 0 & 0 & 3 
            \end{bmatrix*},
        \end{align*}
        \begin{align*}
            \mathbf{B} =  \begin{bmatrix*}[c]
                4 \\
                0 \\
                3 \\
                10 \\
                6  
            \end{bmatrix*}.
        \end{align*}
        \paragraph{}From Theorem 3.5's equation(6), we could calculate $x_1, x_2, x_3, x_4, x_5$:
        $$x_k = \frac{b_k - \sum_{j = k + 1}^{N}a_{kj}x_j }{a_{kk}}, \mathrm{for} \ k = N - 1, N - 2, \dots , 1.$$
        \paragraph{}Solving for $x_5$ in the last equation yields
        $$x_5 = \frac{6}{3} = 2$$
        Then we could obtain:
        \begin{align*}
        x_4 &= \frac{10 - (-1 \times 2)}{-2} = -6 \\
        x_3 &= \frac{3 - (-1 \times (-6) + (-2) \times 2)}{1} = 1 \\
        x_2 &= \frac{0 - (6 \times 1 + 2 \times (-6) + 7 \times 2)}{-2} = 4 \\\
        x_1 &= \frac{4 - ((-1) \times 4 + 2 \times 1 + 2 \times (-6) + (-1) \times 2)}{4} = 5
        \end{align*}
        \paragraph{}And we could calculate $\det(\mathbf{A}) = 4 \times (-2) \times 1 \times (-2) \times 3 = 48$
        \paragraph{4(a).}Consider the two upper-triangular matrices.
        \begin{align*}
            \mathbf{A} = \begin{bmatrix*}[c]
                a_{11} & a_{12} & a_{13} \\
                0 & a_{22} & a_{23} \\
                0 & 0 & a_{33}
            \end{bmatrix*} 
            \mathrm{and} \ 
            \mathbf{B} =  \begin{bmatrix*}[c]
                b_{11} & b_{12} & b_{13} \\
                0 & b_{22} & b_{23} \\
                0 & 0 & b_{33}
            \end{bmatrix*}.
        \end{align*}
        \paragraph{}Show that their product $\mathbf{C} = \mathbf{A} \mathbf{B}$ is also upper triangular.
        \paragraph{Solution:}
        \paragraph{}Using Definition 3.1's equation(7), we could obtain $\mathbf{C}$:
        \begin{align*}
            \mathbf{C} =  \begin{bmatrix*}[c]
            a_{11}b_{11} & a_{11}b_{12} + a_{12}b_{22} & a_{11}b_{13} + a_{12}b_{23} + b_{13}b_{33} \\
            0 & a_{22}b_{22} & a_{22}b_{23} + a_{23}b_{33} \\
            0 & 0 & a_{33}b_{33}
            \end{bmatrix*}
        \end{align*}
        \paragraph{}We could see that $\mathbf{C}$ is upper triangular.
        \paragraph{9.}What will happen if the bisection method is used with the function $f(x) = 1 / (x - 2)$ and
        \begin{enumerate}[(a)]
            \item the interval is $[3, 7]$?
            \item the interval is $[1, 7]$?
        \end{enumerate}
        \paragraph{Solution:}
        \paragraph{(a)}Since $f(3) \cdot f(7) > 0$, the initial condition is not satisfied.
        \paragraph{(b)}After infinite iterations, $c$ will be infinitely close to 2, but $f(x)$ is undefined at 2.
        \paragraph{11.} Suppose that the bisection method is used to find a zero of $f(x)$ in the interval $[2, 7]$. How many times must this interval be bisected to guarantee that the approximation $c_N$ has an accuracy of $5 \times 10^{-9}$?
        \paragraph{Solution:}
        \paragraph{}Use equation(15) to get the number N, where $a = 2, b = 7, \delta = 5 \times 10^{-9}.$
        \begin{align*}
        N &= \mathrm{int}\left(\frac{\ln{(b-a)}-\ln{(\delta)}}{\ln{(2)}}\right) \\
        &= \mathrm{int}\left(\frac{\ln{(7-2)}-\ln{(5 \times 10^{-9})}}{\ln{(2)}}\right) \\
        &= \mathrm{int}\left(29.89735285\right) \\
        &= 29
        \end{align*}
        \subsection*{Page 85}
        \paragraph{4.}Let $f(x) = x^3 - 3x -2$.
        \begin{enumerate}[(a)]
            \item Find the Newton-Raphson formula $p_k = g(p_{k-1})$.
            \item Start with $p_0 = 2.1$ and find $p_1, p_2, p_3$ and $p_4$.
            \item Is the sequence converging quadratically or linearly?
        \end{enumerate}
        \paragraph{Solution:}
        \paragraph{(a)}According to Theorem 2.5,
        \begin{align*}
        g(x) &= x - \frac{f(x)}{f'(x)} \\
        &= x - \frac{x^3 - 3x -2}{3x^2 - 3} \\
        &= \frac{2x^3 + 2}{3x^2 - 3}
        \end{align*}
        \paragraph{}Thus,
        $$p_k = g(p_{k-1}) = \frac{2p_{k-1}^3 + 2}{3p_{k-1}^2 - 3}$$
        \paragraph{(b)}We could use the formula above to calculate:
        $$p_1 = g(p_0) = \frac{2 \times 2.1^3 + 2}{3 \times 2.1^2 - 3} = 2.0060606061$$
        $$p_2 = g(p_1) = \frac{2 \times 2.0060606061^3 + 2}{3 \times 2.0060606061^2 - 3} = 2.0000243398$$
        $$p_3 = g(p_2) = \frac{2 \times 2.0000243398^3 + 2}{3 \times 2.0000243398^2 - 3} = 2.0000000004$$
        $$p_4 = g(p_3) = \frac{2 \times 2.0000000004^3 + 2}{3 \times 2.0000000004^2 - 3} = 2.0000000000$$        
        \paragraph{(c)}Assume $R = 2$ and use equation(19) to examine the quadratical convergence, We find that $A \approx \frac{2}{3}$.
        \paragraph{}Thus the sequence converging quadratically.
        \subsection*{Page 86}
        \paragraph{8.}Use the secant method and formula (27) and compute the next two iterations $p_2$ and $p_3$. Let $f(x) = x^2 - 2x - 1$. Start with $p_0 = 2.6$ and $p_1 = 2.5$. 
        \paragraph{Solution:}
        \begin{align*}
        p_2 &= p_1 - \frac{f(p_1)(p_1 - p_0)}{f(p_1) - f(p_0)} \\
        &= 2.5 - \frac{(2.5^2 - 2 \times 2.5 -1)(2.5 - 2.6)}{(2.5^2 - 2 \times 2.5 -1)(2.6^2 - 2 \times 2.6 -1)} \\ 
        &= 2.41935484
        \end{align*}
        \begin{align*}
        p_3 &= p_2 - \frac{f(p_2)(p_2 - p_1)}{f(p_2) - f(p_1)} \\
        &= 2.41935484 - \frac{(2.41935484^2 - 2 \times 2.41935484 -1)(2.41935484 - 2.5)}{(2.41935484^2 - 2 \times 2.41935484 -1)(2.5^2 - 2 \times 2.5 -1)} \\ 
        &= 2.41436464
        \end{align*}
\end{document}
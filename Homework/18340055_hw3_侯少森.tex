\documentclass{article}  % 选择模版,这里是使用Latex自带的article模版
    \title{Numerical Computing Homework 3}
    \author{Shaosen Hou 18340055}
    \usepackage{multirow}   % 插入表格用到的宏包
    \usepackage{enumerate}
    \usepackage{threeparttable}
    \usepackage{amsmath,bm}
    \usepackage{amsfonts}
    \usepackage[noend]{algpseudocode}
    \usepackage{subcaption}
    \usepackage[english]{babel}	
    \usepackage{paralist}	
    \usepackage[lowtilde]{url}
    \usepackage{listings}
    \usepackage{color}
    \usepackage{hyperref}
    \usepackage{mathtools}
\begin{document} 
    \maketitle
        \subsection*{Page 124} 
        \paragraph{3.}Solve the upper-triangular system and find the value of the determinant of the coefficient matrix.
        \begin{align*}
            4x_1 - x_2 + 2x_3 + 2x_4 - x_5 &= 4 \\
            -2x_2 + 6x_3 + 2x_4 + 7x_5 &= 0 \\
            x_3 - x_4 - 2x_5 &= 3 \\
            -2x_4 - x_5 &= 10 \\
            3x_5 &= 6 \\
        \end{align*}
        \paragraph{Solution:}
        \paragraph{}We find that all the diagonal elements are non-zero. So we could solve the upper-triangular system by the method of back substitution.
        \paragraph{}The coefficient matrix $\bm{A}$ and the matrix $\bm{B}$ are: 
        \begin{align*}
            \bm{A=}  \begin{bmatrix*}[r]
                4 & -1 & 2 & 2 & -1 \\
                0 & -2 & 6 & 2 & 7 \\
                0 & 0 & 1 & -1 & -2 \\
                0 & 0 & 0 & -2 & -1 \\
                0 & 0 & 0 & 0 & 3 
            \end{bmatrix*},
        \end{align*}
        \begin{align*}
            \bm{B=}  \begin{bmatrix*}[r]
                4 \\
                0 \\
                3 \\
                10 \\
                6  
            \end{bmatrix*}.
        \end{align*}
        \paragraph{}From Theorem 3.5's equation(6), we could calculate $x_1, x_2, x_3, x_4, x_5$:
        $$x_k = \frac{b_k - \sum_{j = k + 1}^{N}a_{kj}x_j }{a_{kk}}, \mathrm{for} \ k = N - 1, N - 2, \dots , 1.$$
        \paragraph{}Solving for $x_5$ in the last equation yields
        $$x_5 = \frac{6}{3} = 2$$
        Then we could obtain:
        \begin{align*}
        x_4 &= \frac{10 - (-1 \times 2)}{-2} = -6 \\
        x_3 &= \frac{3 - (-1 \times (-6) + (-2) \times 2)}{1} = 1 \\
        x_2 &= \frac{0 - (6 \times 1 + 2 \times (-6) + 7 \times 2)}{-2} = 4 \\\
        x_1 &= \frac{4 - ((-1) \times 4 + 2 \times 1 + 2 \times (-6) + (-1) \times 2)}{4} = 5
        \end{align*}
        \paragraph{}And we could calculate $\det(\bm{A}) = 4 \times (-2) \times 1 \times (-2) \times 3 = 48$
        \paragraph{4(a).}Consider the two upper-triangular matrices.
        \begin{align*}
            \bm{A=} \begin{bmatrix*}[c]
                a_{11} & a_{12} & a_{13} \\
                0 & a_{22} & a_{23} \\
                0 & 0 & a_{33}
            \end{bmatrix*} 
            \mathrm{and} \ 
            \bm{B=}  \begin{bmatrix*}[c]
                b_{11} & b_{12} & b_{13} \\
                0 & b_{22} & b_{23} \\
                0 & 0 & b_{33}
            \end{bmatrix*}.
        \end{align*}
        \paragraph{}Show that their product $\bm{C = AB}$ is also upper triangular.
        \paragraph{Solution:}
        \paragraph{}Using Definition 3.1's equation(7), we could obtain $\bm{C}$:
        \begin{align*}
            \bm{C=}  \begin{bmatrix*}[c]
            a_{11}b_{11} & a_{11}b_{12} + a_{12}b_{22} & a_{11}b_{13} + a_{12}b_{23} + b_{13}b_{33} \\
            0 & a_{22}b_{22} & a_{22}b_{23} + a_{23}b_{33} \\
            0 & 0 & a_{33}b_{33}
            \end{bmatrix*}
        \end{align*}
        \paragraph{}We could see that $\bm{C}$ is upper triangular.
        \subsection*{Page 137}
        \paragraph{1.}Show that $\bm{AX = B}$ is equivalent to the upper-triangular system $\bm{UX = Y}$ and find the solution.
        \begin{align*}
            2x_1 + 4x_2 - 6x_3 = -4& & 2x_1 + 4x_2 - 6_x3 = -4&\\
            x_1 + 5x_2 + 3x_3 = 10& & 3x_2 + 6x_3 = 12&\\
            x_1 + 3x_2 + 2x_3 = 5& & 3x_3 = 3&
        \end{align*}
        \paragraph{Solution:}
        \paragraph{}The augmented matrix is 
        \begin{equation*}       %开始数学环境
            \left[                 %左括号
              \begin{array}{rrr|r}   %该矩阵一共3列,每一列都居中放置
                2 & 4 & -6 & -4 \\  %第一行元素
                1 & 5 & 3 & 10 \\  %第二行元素
                1 & 3 & 2 & 5
              \end{array}
            \right]                 %右括号
        \end{equation*}
        \paragraph{}The first row is used to eliminate elements in the first column below the diagonal.The result after elimination is
        \begin{equation*}       %开始数学环境
            \left[                 %左括号
              \begin{array}{rrr|r}   %该矩阵一共3列,每一列都居中放置
                2 & 4 & -6 & -4 \\  %第一行元素
                0 & 3 & 6 & 12 \\  %第二行元素
                0 & 1 & 5 & 7
              \end{array}
            \right]                 %右括号
        \end{equation*}
        \paragraph{}The second row is used to eliminate elements in the second column that lie below the
        diagonal.The result after elimination is
        \begin{equation*}       %开始数学环境
            \left[                 %左括号
              \begin{array}{rrr|r}   %该矩阵一共3列,每一列都居中放置
                2 & 4 & -6 & -4 \\  %第一行元素
                0 & 3 & 6 & 12 \\  %第二行元素
                0 & 0 & 3 & 3
              \end{array}
            \right]                 %右括号
        \end{equation*}
        \paragraph{}Thus, $\bm{AX = B}$ is equivalent to the upper-triangular system $\bm{UX = Y}$.
        \paragraph{}The back-substitution algorithm can be used to solve the system, and we get
        \begin{equation*}       %开始数学环境
            \bm{X=}  \left[                 %左括号
              \begin{array}{r}   %该矩阵一共3列,每一列都居中放置
                -3 \\ %第一行元素
                2 \\  %第二行元素
                1 \\ 
              \end{array}
            \right]                 %右括号
        \end{equation*}
        \subsection*{Page 138}
        \paragraph{9.}Show that $\bm{AX = B}$ is equivalent to the upper-triangular system $\bm{UX = Y}$ and find the solution.
        \begin{align*}
            2x_1 + 4x_2 - 4x_3 + 0x_4 = 12& & 2x_1 + 4x_2 - 4x_3 + 0x_4 = 12 &\\
            x_1 + 5x_2 - 5x_3 - 3x_4 = 18& & 3x_2 - 3x_3 - 3x_4 = 12 &\\
            2x_1 + 3x_2 + x_3 + 3x_4 = 8& & 4x_3 + 2x_4 = 0 &\\
            x_1 + 4x_2 - 2x_3 + 2x_4 = 8& & 3x_4 = -6 &
        \end{align*}
        \paragraph{Solution:}
        \paragraph{}The augmented matrix is 
        \begin{equation*}       %开始数学环境
            \left[                 %左括号
              \begin{array}{rrrr|r}   %该矩阵一共3列,每一列都居中放置
                2 & 4 & -4 & 0 & 12 \\  %第一行元素
                1 & 5 & -5 & -3 & 18 \\  %第二行元素
                2 & 3 & 1 & 3 & 8 \\
                1 & 4 & -2 & 2 & 8
              \end{array}
            \right]                 %右括号
        \end{equation*}
        \paragraph{}The first row is used to eliminate elements in the first column below the diagonal.The result after elimination is
        \begin{equation*}       %开始数学环境
            \left[                 %左括号
              \begin{array}{rrrr|r}   %该矩阵一共3列,每一列都居中放置
                2 & 4 & -4 & 0 & 12 \\
                0 & 3 & -3 & -3 & 12 \\
                0 & -1 & 5 & 3 & -4 \\
                0 & 2 & -4 & 2 & 2 
              \end{array}
            \right]                 %右括号
        \end{equation*}
        \paragraph{}The second row is used to eliminate elements in the second column that lie below the
        diagonal.The result after elimination is
        \begin{equation*}       %开始数学环境
            \left[                 %左括号
              \begin{array}{rrrr|r}   %该矩阵一共3列,每一列都居中放置
                2 & 4 & -4 & 0 & 12 \\
                0 & 3 & -3 & -3 & 12 \\
                0 & 0 & 4 & 2 & 0 \\
                0 & 0 & -6 & 0 & -6
              \end{array}
            \right]                 %右括号
        \end{equation*}
        \paragraph{}The third row is used to eliminate elements in the third column that lie below the
        diagonal.The result after elimination is
        \begin{equation*}       %开始数学环境
            \left[                 %左括号
              \begin{array}{rrrr|r}   %该矩阵一共3列,每一列都居中放置
                2 & 4 & -4 & 0 & 12 \\
                0 & 3 & -3 & -3 & 12 \\
                0 & 0 & 4 & 2 & 0 \\
                0 & 0 & 0 & 3 & -6
              \end{array}
            \right]                 %右括号
        \end{equation*}
        \paragraph{}Thus, $\bm{AX = B}$ is equivalent to the upper-triangular system $\bm{UX = Y}$.
        \paragraph{}The back-substitution algorithm can be used to solve the system, and we get
        \begin{equation*}       %开始数学环境
            \bm{X=}  \left[                 %左括号
              \begin{array}{r}   %该矩阵一共3列,每一列都居中放置
                2 \\ %第一行元素
                3 \\  %第二行元素
                1 \\ 
                -2 \\
              \end{array}
            \right]                 %右括号
        \end{equation*}
        \subsection*{Page 153}
        \paragraph{1(a).}$ \mathrm{Solve}$ $\bm{LY = B}, \bm{UX = Y}$, $\mathrm{and}$ verify that $\bm{B = AX}$ $\mathrm{for}$ $\bm{B} = [-4 \ 10 \ 5]'$, where $\bm{A = LU}$ is
        \begin{equation*}       %开始数学环境
            \left[                 %左括号
              \begin{array}{rrr}   %该矩阵一共3列,每一列都居中放置
                2 & 4 & -6 \\
                1 & 5 & 3 \\
                1 & 3 & 2 
              \end{array}
            \right]  \bm=               %右括号
            \left[                 %左括号
              \begin{array}{rrr}   %该矩阵一共3列,每一列都居中放置
                1 & 0 & 0 \\
                1/2 & 1 & 0 \\
                1/2 & 1/3 & 1 
              \end{array}
            \right]
            \left[                 %左括号
              \begin{array}{rrr}   %该矩阵一共3列,每一列都居中放置
                2 & 4 & -6 \\
                0 & 3 & 6 \\
                0 & 0 & 3 
              \end{array}
            \right]
        \end{equation*}
        \paragraph{Solution:}
        \paragraph{}Use the forward-substitution method to solve $\bm{LY = B}$:
        \begin{equation*}       %开始数学环境
            \left[                 %左括号
              \begin{array}{rrr|r}   %该矩阵一共3列,每一列都居中放置
                1 & 0 & 0 & -4\\
                1/2 & 1 & 0 & 10\\
                1/2 & 1/3 & 1 & 5
              \end{array}
            \right]                 %右括号
        \end{equation*}
        \paragraph{}And we obtain
        \begin{equation*}       %开始数学环境
            \bm{Y =} \left[                 %左括号
              \begin{array}{r}   %该矩阵一共3列,每一列都居中放置
                -4 \\
                12 \\
                3
              \end{array}
            \right]                 %右括号
        \end{equation*}
        \paragraph{}Next write the augmented matrix $\bm{UX=Y}$:
        \begin{equation*}       %开始数学环境
            \left[                 %左括号
              \begin{array}{rrr|r}   %该矩阵一共3列,每一列都居中放置
                2 & 4 & -6 & -4\\
                0 & 3 & 6 & 12\\
                0 & 0 & 3 & 3
              \end{array}
            \right]                 %右括号
        \end{equation*}
        \paragraph{}And we obtain
        \begin{equation*}       %开始数学环境
            \bm{X =} \left[                 %左括号
              \begin{array}{r}   %该矩阵一共3列,每一列都居中放置
                -3 \\
                2 \\
                1
              \end{array}
            \right]                 %右括号
        \end{equation*}
        \paragraph{}Thus, 
        \begin{equation*}       %开始数学环境
            \bm{AX=} \left[                 %左括号
              \begin{array}{rrr}   %该矩阵一共3列,每一列都居中放置
                2 & 4 & -6 \\
                1 & 5 & 3 \\
                1 & 3 & 2 
              \end{array}
            \right]                 %右括号
            \left[                 %左括号
              \begin{array}{r}   %该矩阵一共3列,每一列都居中放置
                -3 \\
                2 \\
                1
              \end{array}
            \right] \bm=                 %右括号
            \left[                 %左括号
              \begin{array}{r}   %该矩阵一共3列,每一列都居中放置
                -4 \\
                10 \\
                5
              \end{array}
            \right] \bm{=B}
        \end{equation*}
        \paragraph{4.}Find the triangular factorization $\bm{A = LU}$ for the matrices.
        \begin{enumerate}[(a)]
            \item \begin{equation*}       %开始数学环境
                \left[                 %左括号
                  \begin{array}{rrr}   %该矩阵一共3列,每一列都居中放置
                    4 & 2 & 1 \\
                    2 & 5 & -2 \\
                    1 & -2 & 7
                  \end{array}
                \right]                 %右括号
            \end{equation*}
            \item \begin{equation*}       %开始数学环境
                \left[                 %左括号
                  \begin{array}{rrr}   %该矩阵一共3列,每一列都居中放置
                    1 & -2 & 7 \\
                    4 & 2 & 1 \\
                    2 & 5 & -2 
                  \end{array}
                \right]                 %右括号
            \end{equation*}
        \end{enumerate}
        \paragraph{Solution:}
        \paragraph{(a)}
        % \begin{equation*}       %开始数学环境
            \begin{align*}
            \left[                 %左括号
              \begin{array}{rrr}   %该矩阵一共3列,每一列都居中放置
                4 & 2 & 1 \\
                2 & 5 & -2 \\
                1 & -2 & 7
              \end{array}
            \right]  &\bm=               %右括号
            \left[                 %左括号
              \begin{array}{rrr}   %该矩阵一共3列,每一列都居中放置
                4 & 2 & 1 \\
                1/2 & 4 & -5/2 \\
                1/4 & -5/8 & 83/16
              \end{array}
            \right] \\ &\bm=
            \left[                 %左括号
              \begin{array}{rrr}   %该矩阵一共3列,每一列都居中放置
                1 & 0 & 0 \\
                1/2 & 1 & 0 \\
                1/4 & -5/8 & 1
              \end{array}
            \right]
            \left[                 %左括号
              \begin{array}{rrr}   %该矩阵一共3列,每一列都居中放置
                4 & 2 & 1 \\
                0 & 4 & -5/2 \\
                0 & 0 & 83/16
              \end{array}
            \right]
        \end{align*}
        % \end{equation*}
        \paragraph{(b)}
        % \begin{equation*}       %开始数学环境
            \begin{align*}
            \left[                 %左括号
              \begin{array}{rrr}   %该矩阵一共3列,每一列都居中放置
                1 & -2 & 7 \\
                4 & 2 & 1 \\
                2 & 5 & -2 
              \end{array}
            \right]  &\bm=               %右括号
            \left[                 %左括号
              \begin{array}{rrr}   %该矩阵一共3列,每一列都居中放置
                1 & -2 & 7 \\
                4 & 10 & -27 \\
                2 & 9/10 & 83/10
              \end{array}
            \right] \\ &\bm=
            \left[                 %左括号
              \begin{array}{rrr}   %该矩阵一共3列,每一列都居中放置
                1 & 0 & 0 \\
                4 & 1 & 0 \\
                2 & 9/10 & 1
              \end{array}
            \right]
            \left[                 %左括号
              \begin{array}{rrr}   %该矩阵一共3列,每一列都居中放置
                1 & -2 & 7 \\
                0 & 10 & -27 \\
                0 & 0 & 83/10
              \end{array}
            \right]
        \end{align*}
        % \end{equation*}
        \paragraph{6.}Find the triangular factorization $\bm{A = LU}$ for the matrix
        \begin{equation*}       %开始数学环境
            \left[                 %左括号
              \begin{array}{rrrr}   %该矩阵一共3列,每一列都居中放置
                1 & 1 & 0 & 4 \\
                2 & -1 & 5 & 0 \\
                5 & 2 & 1 & 2 \\
                -3 & 0 & 2 & 6
              \end{array}
            \right]                 %右括号
        \end{equation*}
        \paragraph{Solution:}
        \begin{align*}
            \left[                 %左括号
              \begin{array}{rrrr}   %该矩阵一共3列,每一列都居中放置
                1 & 1 & 0 & 4 \\
                2 & -1 & 5 & 0 \\
                5 & 2 & 1 & 2 \\
                -3 & 0 & 2 & 6 
              \end{array}
            \right]  &\bm=               %右括号
            \left[                 %左括号
              \begin{array}{rrrr}   %该矩阵一共3列,每一列都居中放置
                1 & 1 & 0 & 4 \\
                2 & -3 & 5 & -8 \\
                5 & 1 & -4 & -10 \\
                -3 & -1 & -7/4 & -15/2
              \end{array}
            \right] \\ &\bm=
            \left[                 %左括号
              \begin{array}{rrrr}   %该矩阵一共3列,每一列都居中放置
                1 & 0 & 0 & 0 \\
                2 & 1 & 0 & 0 \\
                5 & 1 & 1 & 0 \\
                -3 & -1 & -7/4 & 1
              \end{array}
            \right]
            \left[                 %左括号
              \begin{array}{rrrr}   %该矩阵一共3列,每一列都居中放置
                1 & 1 & 0 & 4 \\
                0 & -3 & 5 & -8 \\
                0 & 0 & -4 & -10 \\
                0 & 0 & 0 & -15/2
              \end{array}
            \right]
        \end{align*}
        \subsection*{Page 165}
        \paragraph{1.}
        \begin{align*}
            4x - y &= 15 \\
            x + 5y &= 9
        \end{align*}
        \begin{enumerate}
            \item[(a)] 
            Start with $\bm{P}_0 = \bm{0}$ and use Jacobi iteration to find $\bm{P}_k$ for $k = 1, 2, 3$. Will Jacobi iteration converge to the solution?
            \item[(b)]
            Start with $\bm{P}_0 = \bm{0}$ and use Gauss-Seidel iteration to find $\bm{P}_k$ for $k = 1, 2, 3$. Will Gauss-Seidel iteration converge to the solution?
        \end{enumerate}
        \paragraph{Solution:}
        \paragraph{(a)}These equations can be written in the form
        \begin{align*}
            x &= \frac{15 + y}{4} \\
            y &= \frac{9 - x}{5}
        \end{align*}
        \paragraph{}This suggests the following Jacobi iterative process:
        \begin{align*}
            x_{k+1} &= \frac{15 + y_k}{4} \\
            y_{k+1} &= \frac{9 - x_k}{5}
        \end{align*}
        \paragraph{}Substitute $x_0 = 0, y_0 = 0$ into the right-hand side of each equation to obtain the new values
        \begin{align*}
            x_{1} &= \frac{15 + y_0}{4} = \frac{15}{4} = 3.75 \\
            y_{1} &= \frac{9 - x_0}{5} = \frac{9}{5} = 1.8
        \end{align*}
        \paragraph{}Similarly, we could obtain 
        \begin{align*}
            x_{2} &= \frac{15 + y_1}{4} = \frac{15 + 1.8}{4} = 4.2 \\
            y_{2} &= \frac{9 - x_1}{5} = \frac{9 - 3.75}{5} = 1.05
        \end{align*}
        \paragraph{}and
        \begin{align*}
            x_{3} &= \frac{15 + y_2}{4} = \frac{15 + 1.05}{4} = 4.0125 \\
            y_{3} &= \frac{9 - x_2}{5} = \frac{9 - 4.2}{5} = 0.96
        \end{align*}
        \paragraph{}Thus, $\bm{P}_1 = (3.75, 1.8), \bm{P}_2 = (4.2, 1.05), \bm{P}_3 = (4.0125, 0.96)$
        \paragraph{}The coefficient matrix of the linear system is strictly diagonally dominant because
        \paragraph{}
        \renewcommand\tabcolsep{12.0pt} % 调整表格列间的宽度
        \begin{threeparttable} % 需要加载 threeparttable 包
            \begin{tabular}{cccccc} 
             & & & & In row 1: & $|4| > |-1|$ \\
             & & & & In row 2: & $|5| > |1|$
            \end{tabular} 
        \end{threeparttable}
        \paragraph{}According to \textbf{Theorem 3.15}, Jacobi iteration will converge to the solution, which is (4, 1).
        \paragraph{(b)}These equations can be written in the form
        \begin{align*}
            x &= \frac{15 + y}{4} \\
            y &= \frac{9 - x}{5}
        \end{align*}
        \paragraph{}This suggests the following Gauss-Seidel iterative process:
        \begin{align*}
            x_{k+1} &= \frac{15 + y_k}{4} \\
            y_{k+1} &= \frac{9 - x_{k+1}}{5}
        \end{align*}
        \paragraph{}Substitute $x_0 = 0, y_0 = 0$ into the right-hand side of each equation to obtain the new values
        \begin{align*}
            x_{1} &= \frac{15 + y_0}{4} = \frac{15}{4} = 3.75 \\
            y_{1} &= \frac{9 - x_1}{5} = \frac{9 - 3.75}{5} = 1.05
        \end{align*}
        \paragraph{}Similarly, we could obtain 
        \begin{align*}
            x_{2} &= \frac{15 + y_1}{4} = \frac{15 + 1.05}{4} = 4.0125 \\
            y_{2} &= \frac{9 - x_2}{5} = \frac{9 - 4.0125}{5} = 0.9975
        \end{align*}
        \paragraph{}and
        \begin{align*}
            x_{3} &= \frac{15 + y_2}{4} = \frac{15 + 0.9975}{4} = 3.999375 \\
            y_{3} &= \frac{9 - x_3}{5} = \frac{9 - 3.999375}{5} = 1.000125‬
        \end{align*}
        \paragraph{}Thus, $\bm{P}_1 = (3.75, 1.05), \bm{P}_2 = (4.0125, 0.9975), \bm{P}_3 = (3.999375, 1.000125‬)$
        \paragraph{}The coefficient matrix of the linear system is strictly diagonally dominant because
        \paragraph{}
        \renewcommand\tabcolsep{12.0pt} % 调整表格列间的宽度
        \begin{threeparttable} % 需要加载 threeparttable 包
            \begin{tabular}{cccccc} 
             & & & & In row 1: & $|4| > |-1|$ \\
             & & & & In row 2: & $|5| > |1|$
            \end{tabular} 
        \end{threeparttable}
        \paragraph{}According to \textbf{Theorem 3.15}, Gauss-Seidel iteration will converge to the solution, which is (4, 1).
        \paragraph{3.}
        \begin{align*}
            -x + 3y &= 1 \\
            6x - 2y &= 2
        \end{align*}
        \begin{enumerate}
            \item[(a)] 
            Start with $\bm{P}_0 = \bm{0}$ and use Jacobi iteration to find $\bm{P}_k$ for $k = 1, 2, 3$. Will Jacobi iteration converge to the solution?
            \item[(b)]
            Start with $\bm{P}_0 = \bm{0}$ and use Gauss-Seidel iteration to find $\bm{P}_k$ for $k = 1, 2, 3$. Will Gauss-Seidel iteration converge to the solution?
        \end{enumerate}
        \paragraph{Solution:}
        \paragraph{(a)}Similar to \textbf{Exercise 1(a)}, we could obtain 
        \begin{align*}
            x_{k+1} &= 3y_k - 1 \\
            y_{k+1} &= \frac{6x_k - 2}{2}
        \end{align*}
        \paragraph{}Thus, we can similarly get
        \begin{align*}
            &\bm{P}_1 = (-1, -1),  \\
            &\bm{P}_2 = (-4, -4),  \\
            &\bm{P}_3 = (-13, -13‬).
        \end{align*}
        \paragraph{}The coefficient matrix of the linear system is not strictly diagonally dominant because
        \paragraph{}
        \renewcommand\tabcolsep{12.0pt} % 调整表格列间的宽度
        \begin{threeparttable} % 需要加载 threeparttable 包
            \begin{tabular}{cccccc} 
             & & & & In row 1: & $|-1| < |3|$ \\
             & & & & In row 2: & $|-2| < |6|$
            \end{tabular} 
        \end{threeparttable}
        \paragraph{}According to \textbf{Theorem 3.15}, Jacobi iteration will not converge to the solution.
        \paragraph{(b)}Similar to \textbf{Exercise 1(b)}, we could obtain 
        \begin{align*}
            x_{k+1} &= 3y_k - 1 \\
            y_{k+1} &= \frac{6x_{k+1} - 2}{2}
        \end{align*}
        \paragraph{}Thus, we can similarly get
        \begin{align*}
            &\bm{P}_1 = (-1, -4),  \\
            &\bm{P}_2 = (-13, -40),  \\
            &\bm{P}_3 = (-121, -364‬).
        \end{align*}
        \paragraph{}The coefficient matrix of the linear system is not strictly diagonally dominant because
        \paragraph{}
        \renewcommand\tabcolsep{12.0pt} % 调整表格列间的宽度
        \begin{threeparttable} % 需要加载 threeparttable 包
            \begin{tabular}{cccccc} 
             & & & & In row 1: & $|-1| < |3|$ \\
             & & & & In row 2: & $|-2| < |6|$
            \end{tabular} 
        \end{threeparttable}
        \paragraph{}According to \textbf{Theorem 3.15}, Gauss-Seidel iteration will not converge to the solution.
        \paragraph{5.}
        \begin{align*}
            5x - y + z = 10 &\\
            2x + 8y - z = 11 &\\
            -x + y + 4z = 3 &
        \end{align*}
        \begin{enumerate}
            \item[(a)] 
            Start with $\bm{P}_0 = \bm{0}$ and use Jacobi iteration to find $\bm{P}_k$ for $k = 1, 2, 3$. Will Jacobi iteration converge to the solution?
            \item[(b)]
            Start with $\bm{P}_0 = \bm{0}$ and use Gauss-Seidel iteration to find $\bm{P}_k$ for $k = 1, 2, 3$. Will Gauss-Seidel iteration converge to the solution?
        \end{enumerate}
        \paragraph{Solution:}
        \paragraph{(a)}Similar to \textbf{Exercise 1(a)}, we could obtain 
        \begin{align*}
            x_{k+1} &= \frac{10 + y_k - z_k}{5} \\
            y_{k+1} &= \frac{11 -2x_k + z_k}{8} \\
            z_{k+1} &= \frac{3 + x_k - y_k}{4}
        \end{align*}
        \paragraph{}Thus, we can similarly get
        \begin{align*}
            &\bm{P}_1 = (2, 1.375, 0.75),  \\
            &\bm{P}_2 = (2.125, 0.96875, 0.90625),  \\
            &\bm{P}_3 = (2.0125, 0.95703125, 1.0390625).
        \end{align*}
        \paragraph{}The coefficient matrix of the linear system is strictly diagonally dominant because
        \paragraph{}
        \renewcommand\tabcolsep{12.0pt} % 调整表格列间的宽度
        \begin{threeparttable} % 需要加载 threeparttable 包
            \begin{tabular}{cccccc} 
             & & & & In row 1: & $|5| > |1| + |1|$ \\
             & & & & In row 2: & $|8| > |2| + |-1|$ \\
             & & & & In row 3: & $|4| > |-1| + |1|$ 
            \end{tabular} 
        \end{threeparttable}
        \paragraph{}According to \textbf{Theorem 3.15}, Jacobi iteration will converge to the solution, which is (2, 1, 1).
        \paragraph{(b)}Similar to \textbf{Exercise 1(b)}, we could obtain 
        \begin{align*}
            x_{k+1} &= \frac{10 + y_k - z_k}{5} \\
            y_{k+1} &= \frac{11 -2x_{k+1} + z_{k+1}}{8} \\
            z_{k+1} &= \frac{3 + x_{k+1} - y_{k+1}}{4}
        \end{align*}
        \paragraph{}Thus, we can similarly get
        \begin{align*}
            &\bm{P}_1 = (2, 0.875, 1.03125),  \\
            &\bm{P}_2 = (1.96875, 1.01171875, 0.9892578125),  \\
            &\bm{P}_3 = (2.0044921875, 0.9975341796875, 1.001739501953125).
        \end{align*}
        \paragraph{}The coefficient matrix of the linear system is strictly diagonally dominant because
        \paragraph{}
        \renewcommand\tabcolsep{12.0pt} % 调整表格列间的宽度
        \begin{threeparttable} % 需要加载 threeparttable 包
            \begin{tabular}{cccccc} 
             & & & & In row 1: & $|5| > |1| + |1|$ \\
             & & & & In row 2: & $|8| > |2| + |-1|$ \\
             & & & & In row 3: & $|4| > |-1| + |1|$ 
            \end{tabular} 
        \end{threeparttable}
        \paragraph{}According to \textbf{Theorem 3.15}, Gauss-Seidel iteration will converge to the solution, which is (2, 1, 1).
\end{document}
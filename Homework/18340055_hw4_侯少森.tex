\documentclass{article}  % 选择模版,这里是使用Latex自带的article模版
    \title{Numerical Computing Homework 4}
    \author{Shaosen Hou 18340055}
    \usepackage{multirow}   % 插入表格用到的宏包
    \usepackage{enumerate}
    \usepackage{threeparttable}
    \usepackage{amsmath,bm}
    \usepackage{amsfonts}
    \usepackage[noend]{algpseudocode}
    \usepackage{subcaption}
    \usepackage[english]{babel}	
    \usepackage{paralist}	
    \usepackage[lowtilde]{url}
    \usepackage{listings}
    \usepackage{color}
    \usepackage{hyperref}
    \usepackage{mathtools}
\begin{document}
    \maketitle
        \subsection*{Page 195} 
        \paragraph{1(b).}Let $f(x) = \sin(x)$ and apply Theorem 4.1. Show that if $|x| \le 1$, then the approximation 
        $$\sin(x) \approx x - \frac{x^3}{3!} + \frac{x^5}{5!} - \frac{x^7}{7!} + \frac{x^9}{9!}$$
        has the error bound $|E_9(x)| < 1/10! \le 2.75574 \times 10^{-7}.$
        \paragraph{Solution:}
        \paragraph{}Because $|x| \le 1, |x_0| = |0| \le 1,$ from Theorem 4.1 we could obtain that
        $$|E_9(x)| = |\frac{f^{10}(c)}{10!}(x - 0)^{10}| = \frac{sin(c)}{10!}x^{10} \le \frac{1}{10!} \le 2.75574 \times 10^{-7}$$
        for some value $c = c(x)$ that lies between 0 and 1.
        \paragraph{2(b).}Let $f(x) = \cos(x)$ and apply Theorem 4.1. Show that if $|x| \le 1$, then the approximation 
        $$\cos(x) \approx 1 - \frac{x^2}{2!} + \frac{x^4}{4!} - \frac{x^6}{6!} + \frac{x^8}{8!}$$
        has the error bound $|E_8(x)| < 1/9! \le 2.75574 \times 10^{-6}.$
        \paragraph{Solution:}
        \paragraph{}Because $|x| \le 1, |x_0| = |0| \le 1,$ from Theorem 4.1 we could obtain that
        $$|E_8(x)| = |\frac{f^{9}(c)}{9!}(x - 0)^{9}| = \frac{sin(c)}{9!}x^{9} \le \frac{1}{9!} \le 2.75574 \times 10^{-6}$$
        for some value $c = c(x)$ that lies between 0 and 1.
        \subsection*{Page 205}
        \paragraph{2.}Consider $P(x) = -0.04x^3 + 0.14x^2 - 0.16x + 2.08$, which passes through the four points $(0, 2.08), (1, 2.02), (2, 2.00),$ and $(4, 1.12)$.
        \begin{enumerate}
            \item[(a)]  Find $P(3)$.
            \item[(b)]  Find $P'(3)$.
            \item[(c)]  Find the definite integral of $P(x)$ taken over $[0, 3]$.
            \item[(d)]  Find the extrapolated value $P(4.5)$.
            \item[(e)]  Show how to find the coefficients of $P(x)$.
        \end{enumerate}
        \paragraph{Solution:}
        \paragraph{(a)}We could use Horner's methods to find $P(3)$:
        \begin{align*}
            b_3 &= a_3 = -0.04\\
            b_2 &= a_2 + b_3x = 0.14 + (-0.04) \times 3 = 0.02\\
            b_1 &= a_1 + b_2x = -0.16 + 0.02 \times 3 = -0.1\\
            b_0 &= a_0 + b_1x = 2.08 + (-0.1) \times 3 = 1.78
        \end{align*}
        \paragraph{}The interpolated value is $P(3) = 1.78$.
        \paragraph{(b)}Similarly, we could use Horner's methods to find $P'(3)$:
        \begin{align*}
            d_2 &= 3a_3 = 3 \times (-0.04) = -0.12\\
            d_1 &= 2a_2 + d_2x = 2 \times 0.14 + (-0.12) \times 3 = -0.08\\
            d_0 &= a_1 + d_1x = -0.16 + (-0.08) \times 3 = -0.4
        \end{align*}
        \paragraph{}The numerical derivative is $P'(3) = -0.4$.
        \paragraph{(c)}Similarly, we could use Horner's methods to find $\int_0^3 P(x)\mathrm{d}x$:
        \begin{align*}
            i_4 &= \frac{a_3}{4} = \frac{-0.04}{4} = -0.01\\
            i_3 &= \frac{a_2}{3} + i_4x = \frac{0.14}{3} + (-0.01) \times 3 = 0.0166667\\
            i_2 &= \frac{a_1}{2} + i_3x = \frac{-0.16}{2} + 0.0166667 \times 3 = -0.03\\
            i_1 &= a_0 + i_2x = 2.08 + (-0.03) \times 3 = 1.99\\
            i_0 &= i_1x = 1.99 \times 3 = 5.97
        \end{align*}
        \paragraph{}Hence $I(3) = 5.97$. Similarly, $I(0) = 0$.
        \paragraph{}Therefore, $\int_0^3 P(x)\mathrm{d}x = I(3) - I(0) = 5.97$.
        \paragraph{(d)}We could use Horner's methods to find $P(4.5)$:
        \begin{align*}
            b_3 &= a_3 = -0.04\\
            b_2 &= a_2 + b_3x = 0.14 + (-0.04) \times 4.5 = -0.04\\
            b_1 &= a_1 + b_2x = -0.16 + (-0.04) \times 4.5 = -0.34\\
            b_0 &= a_0 + b_1x = 2.08 + (-0.34) \times 4.5 = 0.55
        \end{align*}
        \paragraph{}The extrapolated value is $P(4.5) = 0.55$.
        \paragraph{(e)}The methods of Chapter 3 can be used to find the coefficients. Assume that $P(x) = A + Bx + Cx_2 + Dx_3;$ then at each value $x = 0, 1, 2,$ and $4$ we get a linear equation involving A, B, C, and D.
        \paragraph{}
        \renewcommand\tabcolsep{6.0pt} % 调整表格列间的宽度
            \begin{threeparttable} % 需要加载 threeparttable 包
                \begin{tabular}{rrrrrrrrrrr} 
                & At $x = 0$: & $A$ & + & 0 & + & 0 & + & 0 & = & 2.08 \\ 
                & At $x = 1$: & $A$ & + & $B$ & + & $C$ & + & $D$ & = & 2.02 \\ 
                & At $x = 2$: & $A$ & + & $2B$ & + & $4C$ & + & $8D$ & = & 2.00 \\ 
                & At $x = 4$: & $A$ & + & $4B$ & + & $16C$ & + & $64D$ & = & 1.12 \\ 
                \end{tabular} 
            \end{threeparttable}
        \paragraph{}The solution is 
        \begin{align*}
            A &=  2.08  \\
            B &=  -0.16  \\
            C &=  0.14  \\
            D &=  -0.04 
        \end{align*}
        \subsection*{Page 217}
        \paragraph{1.}Find the Lagrange polynomials that approximation $f(x) = x^3$.
        \begin{enumerate}
            \item[(a)] 
            Find the linear interpolation polynomial $P_1(x)$ using the nodes $x_0 = -1$ and $x_1 = 0$.
            \item[(b)]
            Find the quadratic interpolation polynomial $P_2(x)$ using nodes $x_0 = -1, x_1 = 0, x_2 = 1$.
            \item[(c)]
            Find the cubic interpolation polynomial $P_3(x)$ using the nodes $x_0 = -1, x_ 1 = 0, x_2 = 1$ and $x_3 = 2$.
            \item[(d)]
            Find the linear interpolation polynomial $P_1(x)$ using the nodes $x_0 = 1$ and $x_1 = 2$.
            \item[(e)]
            Find the quadratic interpolation polynomial $P_2(x)$ using the nodes $x_0 = 0, x_1 = 1$ and $x_2 = 2$.
        \end{enumerate}
        \paragraph{Solution:}
        \paragraph{}We know that a polynomial $P_N(x)$ of degree at most $N$ that passes through the $N + 1$ points $(x_0, y_0), (x_1, y_1), \cdots (x_N, y_N)$ and has the form
        $$P_N(x) = \sum_{k=0}^{N}y_kL_{N,k}(x),$$
        where $L_{N,k}$ is the Lagrange coefficient polynomial based on these nodes:
        $$L_{N, k}(x) = \frac{\prod_{j=0, j \ne k}^N(x - x_j)}{\prod_{j=0, j \ne k}^N(x_k - x_j)}.$$
        \paragraph{(a)}Using $P_N(x)$ with the abscissas $x_0 = -1$ and $x_1 = 0$ and the ordinates $y_0 = (-1)^3= -1$ and $y_1 = 0^3 = 0$ produces
        \begin{align*}
          P_1(x) &= -1\frac{x - 0}{-1 - 0} + 0\frac{x - (-1)}{0 - (-1)}\\
          &= x
        \end{align*}
        \paragraph{(b)}Using $x_0 = -1, x_1 = 0, x_2 = 1$ and $y_0 = (-1)^3 = -1, y_1 = 0^3 = 0, y_2 = 1^3 = 1$ produces
        \begin{align*}
          P_2(x) &= -1\frac{(x - 0)(x - 1)}{(-1 - 0)(-1 - 1)} + 0\frac{(x - (-1))(x - 1)}{(0 - (-1))(0 - 1)}\\
          &+ 1\frac{(x - (-1))(x - 0)}{(1 - (-1))(1 - 0)}\\
          &= -1\frac{x(x-1)}{2} + 0 + \frac{x(x+1)}{2} \\
          &= x
        \end{align*}
        \paragraph{(c)}Using $x_0 = -1, x_1 = 0, x_2 = 1, x_3 = 2$ and $y_0 = (-1)^3 = -1, y_1 = 0^3 = 0, y_2 = 1^3 = 1, y_3 = 2^3 = 8$ produces
        \begin{align*}
          P_3(x) &= -1\frac{(x - 0)(x - 1)(x - 2)}{(-1 - 0)(-1 - 1)(-1 - 2)} \\
          &+ 0\frac{(x - (-1))(x - 1)(x - 2)}{(0 - (-1))(0 - 1)(0 - 2)} \\
          &+ 1\frac{(x - (-1))(x - 0)(x - 2)}{(1 - (-1))(1 - 0)(1 - 2)} \\
          &+ 8\frac{(x - (-1))(x - 0)(x - 1)}{(2 - (-1))(2 - 0)(2 - 1)} \\
          &= -1\frac{x(x - 1)(x - 2)}{-6} + 0 \\
          &+ \frac{x(x + 1)(x - 2)}{-2} \\
          &+ 8\frac{x(x + 1)(x - 1)}{6} \\
          &= x^3
        \end{align*}
        \paragraph{(d)}Same as (a), we could easily obtain $P_1(x)$:
        \begin{align*}
          P_1(x) &= 1\frac{x - 2}{1 - 2} + 8\frac{x - 1}{2 - 1}\\
          &= 7x - 6
        \end{align*}
        \paragraph{(e)}Same as (b), we could easily obtain $P_2(x)$:
        \begin{align*}
          P_2(x) &= 0\frac{(x - 1)(x - 2)}{(0 - 1)(0 - 2)} + 1\frac{(x - 0)(x - 2)}{(1 - 0)(1 - 2)}\\
          &+ 8\frac{(x - 0)(x - 1)}{(2 - 0)(2 - 1)}\\
          &= 0 + 1\frac{x(x-2)}{-1} + \frac{x(x - 1)}{2} \\
          &= 3x^2 - 2x
        \end{align*}
        \paragraph{2.}Let $f(x) = x + 2 / x$.
        \begin{enumerate}
            \item[(a)] 
            Use quadratic Lagrange interpolation based on the nodes $x_0 = 1, x_1 = 2$, and $x_2 = 2.5$ to approximate $f(1.5)$ and $f(1.2)$.
            \item[(b)]
            Use cubic Lagrange interpolation based on the nodes $x_0 = 0.5, x_1 = 1, x_2 = 2, $ and $x_3 = 2.5$ to approximate $f(1.5)$ and $f(1.2)$. 
        \end{enumerate}
        \paragraph{Solution:}
        \paragraph{(a)}Similar to Exercise 1, we could easily obtain $P_2(x)$:
        \begin{align*}
          P_2(x) &= 3\frac{(x - 2)(x - 2.5)}{(1 - 2)(1 - 2.5)} + 3\frac{(x - 1)(x - 2.5)}{(2 - 1)(2 - 2.5)}\\
          &+ 3.3\frac{(x - 1)(x - 2)}{(2.5 - 1)(2.5 - 2)}\\
          &= 0 + 1\frac{x(x-2)}{-1} + \frac{x(x - 1)}{2} \\
          &= 0.4x^2 - 1.2x + 3.8
        \end{align*}
        \paragraph{}Thus,
        \begin{align*}
          f(1.5) &\approx P_2(1.5) = 2.9 \\
          f(1.2) & \approx P_2(1.2) = 2.936
        \end{align*}
        \paragraph{(b)}Similar to Exercise 1, we could easily obtain $P_3(x)$:
        \begin{align*}
          P_3(x) &= -0.8x^3 + 4.8x^2 -8.8x +7.8 \\
        \end{align*}
        \paragraph{}Thus, 
        \begin{align*}
          f(1.5) &\approx P_3(1.5) = 2.7 \\
          f(1.2) & \approx P_3(1.2) = 2.7696
        \end{align*}
        \subsection*{Page 229}
        \paragraph{7.}
        Given that $f(x) = 3\sin^2(\pi x / 6), x = 1.5, 3.5$.
        \begin{table}[h] \label{tab:newton}
        \centering 
        \begin{tabular}{c|c|c}
            \hline
            $k$ & $x_k$ & $f(x_k)$ \\
            \hline
            0 & 0.0 & 0.00 \\
            1 & 1.0 & 0.75 \\
            2 & 2.0 & 2.25 \\
            3 & 3.0 & 3.00 \\
            4 & 4.0 & 2.25 \\
            \hline
        \end{tabular}
        \end{table}
        \begin{enumerate}
        \item[(a)] 
        Compute the divide-difference table for the tabulated function.
        \item[(b)]
        Write down the Newton polynomials $P_1(x), P_2(x), P_3(x)$, and $P_4(x)$.
        \item[(c)]
        Evaluate the Newton polynomials in part (b) at the given values of $x$.
        \item[(d)]
        Compare the values in part (c) with the actual function value $f(x)$.
        \end{enumerate}
        \paragraph{Solution:}
        \paragraph{(a)}The divide-difference table for the tabulated function is as follows:
        \paragraph{}
        \renewcommand\tabcolsep{12.0pt} % 调整表格列间的宽度
            \begin{threeparttable} % 需要加载 threeparttable 包
              %\centering
                \begin{tabular}{c|c|c|c|c|c} 
              \hline
              $x_k$ & $f[x_k]$ & $f[ , ]$  & $f[ , , ]$ & $f[ , , , ]$ & $f[ , , , , ]$ \\
              \hline
              $x_0 = 0$ & 0.00 & 0 & 0 & 0 & 0\\
              \cline{2-2}
              $x_1 = 1$ & 0.75 & 0.75 & 0 & 0 & 0\\
              \cline{2-3}
              $x_2 = 2$ & 2.25 & 1.5 & 0.375 & 0 & 0\\
              \cline{2-4}
              $x_3 = 3$ & 3.00 & 0.75 & -0.375 & -0.25 & 0\\
              \cline{2-5}
              $x_4 = 4$ & 2.25 & -0.75 & -0.75 & -0.125 & 0.03125\\
              \hline
            \end{tabular} 
          \end{threeparttable}
        \paragraph{(b)}The Newton polynomial is
        $$P_N(x) = a_0 + a_1(x - x_0) + \cdots + a_N(x - x_0)(x - x_1)\cdots(x - x_N),$$
        where $a_k = f_[x_0, x_1 \cdots x_k],$ for $k = 0, 1, \cdots N$.
        \paragraph{}Thus, 
        \begin{align*}
          P_1(x) &= 0 + 0.75(x - 0) \\
          P_2(x) &= 0 + 0.75(x - 0) + 0.375(x - 0)(x - 1) \\
          P_3(x) &= 0 + 0.75(x - 0) + 0.375(x - 0)(x - 1) + (-0.25)(x - 0)(x - 1)(x - 2) \\
          P_4(x) &= 0 + 0.75(x - 0) + 0.375(x - 0)(x - 1) + (-0.25)(x - 0)(x - 1)(x - 2) \\
          &+0.03125(x - 0)(x - 1)(x - 2)(x - 3)
        \end{align*}
        \paragraph{(c)}
        \begin{align*}
          P_1(1.5) &= 1.125 & P_1(3.5) &= 2.625\\
          P_2(1.5) &= 1.40625 & P_1(3.5) &= 5.90625\\
          P_3(1.5) &= 1.5 & P_1(3.5) &= 2.625\\
          P_4(1.5) &= 1.517578125 & P_1(3.5) &= 2.830078125
        \end{align*}
        \paragraph{(d)}
        \begin{align*}
          f(1.5) &= 1.5 & f(3.5) &= 2.799038106\\
        \end{align*}
        \paragraph{}$|P_3(1.5) - f(1.5)| = 0$, which is the nearest value between $P_i(1.5)$, $i = 1, 2, 3, 4$.
        \paragraph{}$|P_4(3.5) - f(3.5)| = 0.03104$, which is the nearest value between $P_i(3.5)$, $i = 1, 2, 3, 4$.
\end{document}
\documentclass{article}  % 选择模版,这里是使用Latex自带的article模版
    \title{Numerical Computing Homework 2}
    \author{Shaosen Hou 18340055}
    \usepackage{multirow}   % 插入表格用到的宏包
    \usepackage{enumerate}
    \usepackage{threeparttable}
    \usepackage{amsmath,bm}
    \usepackage{amsfonts}
\begin{document} 
    \maketitle

        \subsection*{Page 50} 
        \paragraph{1.}Determine rigorously if each function has a unique fixed point on the given interval (follow Example 2.3).
        \begin{enumerate}[(a)]
            \item \(g(x) = 1 - x^2 / 4\) on \([0, 1]\)
            \item \(g(x) = 2^{-x}\) on \([0, 1]\)
            \item \(g(x) = 1 / x\) on \([0.5, 5.2]\)
        \end{enumerate}
        \paragraph{Solution:}
        \paragraph{(a)}Clearly, \(g\) $\in$ \(C[0, 1]\). Also, because \(g'(x) = -x/2 <0\) on [0, 1], \(g(x)\) is a decreasing function on [0, 1]; thus its range on [0, 1] is $[\frac{3}{4}, 1] \subseteq [0, 1].$ Thus condition(3) of the Theorem 2.2 is satisfied and $g$ has a fixed point on [0, 1].
        \paragraph{}Finally, if $x$ $\in$ $(0, 1)$, then $|g'(x)| = |-x/2| = x/2 \leq 1/2 < 1$. Thus condition(4) of Theorem 2.2 is satisfied, and $g$ has a unique fixed point in [0, 1].
        \paragraph{(b)}Clearly, \(g\) $\in$ \(C[0, 1]\). Also, because \(g'(x) = -2^{-x}\mathrm{ln}2 <0\) on [0, 1], \(g(x)\) is a decreasing function on [0, 1]; thus its range on [0, 1] is $[\frac{1}{2}, 1] \subseteq [0, 1].$ Thus condition(3) of the Theorem 2.2 is satisfied and $g$ has a fixed point on [0, 1].
        \paragraph{}Finally, if $x$ $\in$ $(0, 1)$, then $|g'(x)| = |-2^{-x}\mathrm{ln}2| = 2^{-x}\mathrm{ln}2 \leq \mathrm{ln}2 < 1$. Thus condition(4) of Theorem 2.2 is satisfied, and $g$ has a unique fixed point in [0, 1].
        \paragraph{(c)}Clearly, \(g\) $\in$ \(C[0.5, 5.2]\). Also, because \(g'(x) = -x^{-2}<0\) on [0.5, 5.2], \(g(x)\) is a decreasing function on [0.5, 5.2]; thus its range on [0.5, 5.2] is $[\frac{5}{26}, 2] \not\subseteq [0.5, 5.2].$ Thus condition(3) of the Theorem 2.2 is not satisfied and $g$ doesn't have a fixed point on [0.5, 5.2].
        \paragraph{}Finally, $g$  doesn't have a unique fixed point in [0.5, 5.2].
        \paragraph{4.}Let $g(x) = x^2 + x - 4$. Can fixed-point iteration be used to find the solution(s) to the equation $x = g(x)$? Why?
        \paragraph{Solution:}
        \paragraph{}Solve the equation $x = x^2 + x - 4$, we can conclude that the fixed points of $g(x)$ are $P_1 = 2$ and $P_2 = -2$. While $|g'(2)| = 5 > 1$ and $|g'(-2)| = 3 > 1$, fixed-point iteration will not converge to $P_1$ and $P_2$.
        \paragraph{}Thus fixed-point iteration cannot be used to find the solution(s) to the equation $x = g(x)$.
        \paragraph{5.}Let $g(x) = x\cos x$. Solve $x = g(x)$ and find all the fixed points of $g$ (there are infinitely many). Can fixed-point iteration be used to find the solution(s) to the equation $x = g(x)$? Why? 
        \paragraph{Solution:}
        \paragraph{}Solve the equation $x = x\cos x$, we can conclude that the fixed points of $g(x)$ are $P = 2k\pi, k \in \mathbb{Z}.$ While $|g'(P)| = |\cos(2k\pi) - 2k\pi\sin(2k\pi)| = |1 - 0| = 1$
        \paragraph{}Thus Theorem 2.3 might not be used to find the solution(s) to the equation $x = g(x)$.            
        \subsection*{Page 61} 
        \paragraph{4.}Start with $[a_0, b_0]$, and use the false position method to compute $c_0, c_1, c_2$ and $c_3$.
        $$e^x - 2 - x = 0, [a_0, b_0] = [-2.4, -1.6] $$
        \paragraph{Solution:}
        \paragraph{}We can use equation(22) to compute $c_0, c_1, c_2, c_3$.
        $$c_n = b_n - \frac{f(b_n)(b_n - a_n)}{f(b_n) - f(a_n)}$$
        \paragraph{}The procedure of false position method is as follows.
        \paragraph{}
        \renewcommand\tabcolsep{10.0pt} % 调整表格列间的宽度
            \begin{threeparttable} % 需要加载 threeparttable 包
                \begin{tabular}{c|c|c|c|c} 
                \hline
                 & Left & & Right & Function value, \\ 
                k & endpoint,$a_k$ & Midpoint,$c_k$ & endpoint,$b_k$ & $f(c_k)$ \\
                \hline
                0 & -2.40000000 & -1.83007818 & -1.60000000 & -0.00952079 \\
                1 & -2.40000000 & -1.84092522 & -1.83007818 & -0.00040423 \\
                2 & -2.40000000 & -1.84138538 & -1.84092522 & -0.00001707 \\
                3 & -2.40000000 & -1.84140480 & -1.84138538 & -0.00000072 \\
                \end{tabular} 
            \end{threeparttable}
        \paragraph{}Thus, $$c_0 = -1.83007818,$$ $$c_1 = -1.84092522,$$ $$c_2 = -1.84138538,$$ $$c_3 = -1.84140480.$$
        \paragraph{9.}What will happen if the bisection method is used with the function $f(x) = 1 / (x - 2)$ and
        \begin{enumerate}[(a)]
            \item the interval is $[3, 7]$?
            \item the interval is $[1, 7]$?
        \end{enumerate}
        \paragraph{Solution:}
        \paragraph{(a)}Since $f(3) \cdot f(7) > 0$, the initial condition is not satisfied.
        \paragraph{(b)}After infinite iterations, $c$ will be infinitely close to 2, but $f(x)$ is undefined at 2.
        \paragraph{11.} Suppose that the bisection method is used to find a zero of $f(x)$ in the interval $[2, 7]$. How many times must this interval be bisected to guarantee that the approximation $c_N$ has an accuracy of $5 \times 10^{-9}$?
        \paragraph{Solution:}
        \paragraph{}Use equation(15) to get the number N, where $a = 2, b = 7, \delta = 5 \times 10^{-9}.$
        \begin{align*}
        N &= \mathrm{int}\left(\frac{\ln{(b-a)}-\ln{(\delta)}}{\ln{(2)}}\right) \\
        &= \mathrm{int}\left(\frac{\ln{(7-2)}-\ln{(5 \times 10^{-9})}}{\ln{(2)}}\right) \\
        &= \mathrm{int}\left(29.89735285\right) \\
        &= 29
        \end{align*}
        \subsection*{Page 85}
        \paragraph{4.}Let $f(x) = x^3 - 3x -2$.
        \begin{enumerate}[(a)]
            \item Find the Newton-Raphson formula $p_k = g(p_{k-1})$.
            \item Start with $p_0 = 2.1$ and find $p_1, p_2, p_3$ and $p_4$.
            \item Is the sequence converging quadratically or linearly?
        \end{enumerate}
        \paragraph{Solution:}
        \paragraph{(a)}According to Theorem 2.5,
        \begin{align*}
        g(x) &= x - \frac{f(x)}{f'(x)} \\
        &= x - \frac{x^3 - 3x -2}{3x^2 - 3} \\
        &= \frac{2x^3 + 2}{3x^2 - 3}
        \end{align*}
        \paragraph{}Thus,
        $$p_k = g(p_{k-1}) = \frac{2p_{k-1}^3 + 2}{3p_{k-1}^2 - 3}$$
        \paragraph{(b)}We could use the formula above to calculate:
        $$p_1 = g(p_0) = \frac{2 \times 2.1^3 + 2}{3 \times 2.1^2 - 3} = 2.0060606061$$
        $$p_2 = g(p_1) = \frac{2 \times 2.0060606061^3 + 2}{3 \times 2.0060606061^2 - 3} = 2.0000243398$$
        $$p_3 = g(p_2) = \frac{2 \times 2.0000243398^3 + 2}{3 \times 2.0000243398^2 - 3} = 2.0000000004$$
        $$p_4 = g(p_3) = \frac{2 \times 2.0000000004^3 + 2}{3 \times 2.0000000004^2 - 3} = 2.0000000000$$        
        \paragraph{(c)}Assume $R = 2$ and use equation(19) to examine the quadratical convergence, We find that $A \approx \frac{2}{3}$.
        \paragraph{}Thus the sequence converging quadratically.
        \subsection*{Page 86}
        \paragraph{8.}Use the secant method and formula (27) and compute the next two iterations $p_2$ and $p_3$. Let $f(x) = x^2 - 2x - 1$. Start with $p_0 = 2.6$ and $p_1 = 2.5$. 
        \paragraph{Solution:}
        \begin{align*}
        p_2 &= p_1 - \frac{f(p_1)(p_1 - p_0)}{f(p_1) - f(p_0)} \\
        &= 2.5 - \frac{(2.5^2 - 2 \times 2.5 -1)(2.5 - 2.6)}{(2.5^2 - 2 \times 2.5 -1)(2.6^2 - 2 \times 2.6 -1)} \\ 
        &= 2.41935484
        \end{align*}
        \begin{align*}
        p_3 &= p_2 - \frac{f(p_2)(p_2 - p_1)}{f(p_2) - f(p_1)} \\
        &= 2.41935484 - \frac{(2.41935484^2 - 2 \times 2.41935484 -1)(2.41935484 - 2.5)}{(2.41935484^2 - 2 \times 2.41935484 -1)(2.5^2 - 2 \times 2.5 -1)} \\ 
        &= 2.41436464
        \end{align*}
\end{document}